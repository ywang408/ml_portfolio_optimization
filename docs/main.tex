\documentclass[a4paper]{article}

\usepackage{fullpage} % Package to use full page
\usepackage{parskip} % Package to tweak paragraph skipping
\usepackage{tikz} % Package for drawing
\usepackage{amsmath}
\usepackage{amsfonts}
\usepackage{hyperref}

\usepackage{setspace}
\doublespacing

\usepackage{natbib}

\title{Classification Methods on Portfolio Optimization}
\author{Eric Wang}
\date{\today}

\renewcommand{\figurename}{Ffigwr}

\begin{document}

\maketitle

\section{Introduction}

\subsection{Mean-Variance Optimization}

We begin with Mean-Variance Optimization(MVO) proposed by \cite{markowitz_portfolio_1952}...

Unfortunately, MVO relies on an accurate estimate of expected return $\mu$ and the covariance matrix $\Sigma$. In real world, its performance is very sensitive to the estimation error and often requires large dataset. Typically, the sample covariance is defined by

$$
\Sigma=\frac{1}{N-1} \mathbf{X X}^{\prime}
$$

where

$$
\mathbf{X}=\left[\begin{array}{ccc}
    R_{11} & \ldots & R_{1 T} \\
    \vdots & \ddots & \vdots \\
    R_{N 1} & \ldots & R_{N T}
    \end{array}\right]-\left[\begin{array}{ccc}
    \bar{R}_1 & \ldots & \bar{R}_1 \\
    \vdots & \ddots & \vdots \\
    \bar{R}_N & \ldots & \bar{R}_N
    \end{array}\right]
$$


\bibliographystyle{chicago}
\bibliography{ref.bib}
\end{document}